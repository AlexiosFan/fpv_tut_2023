\documentclass[12pt]{article}
\usepackage[utf8]{inputenc}
\usepackage{amsmath}
\usepackage{amsfonts}
\usepackage{amssymb}
\usepackage{hyperref}
\usepackage{graphicx}

\title{A brief introduction to predicate logic \footnote{This is unofficial material generated by tutors. 
Please follow the lecture notes for the accurate information. The materials are based on the lecture notes 
of the course Logic(IN2049) by Prof. Esparza and Prof. Nipkow}}
\author{Zixuan Fan, Michael David Kuckuk}
\begin{document}

\maketitle

\section*{Predicates}
\begin{enumerate}
    \item Predicates are logical formulas that contain variables and constants. Variables are usually given by $x, y, z$, whereas constants are integers in context of Mini-Java.
    \item Besides the property of propositional logical formulas, predicates are usually definied with relations.
    $P(x)$ is a predicate describing the property of variable x. $x = 0$, $x + 1 < 2$, and $x + y = 0 \land x < 0$ are all 
    predicates of x.
    \item A predicate may contain more than one variables. The number of the variables in the predicate is
    called the arity of the predicate. For example $P(x, y, z)$ has the arity of 3.
    \item With the existence of $<$ relation, we can also apply the related rules to simplify the logical formula. For example
    \begin{align*}
        &\; (a > 0 \lor x + y > 0) \land (a \leq 0 \lor x + y > 0)\\
        \equiv& \; (a > 0 \land a \leq 0) \lor x + y > 0\\
        \equiv& \; \text{true} \lor x + y > 0\\
        \equiv& \; x + y > 0\\
    \end{align*}
    Equivalent formulas can be exchanged as one sees fit. This includes (but is not limited to) rules 
     like \textbf{reflexivity, transitivity, associativity} etc. whenever applicable
    \begin{align*}
        \exists a. b < a \land a < c \Longrightarrow b < c
    \end{align*}
\end{enumerate}



\section*{Quantifiers}
\begin{enumerate}
    \item There are two different quantifiers, the universal quantifier $\forall$ and the existential quantifier $\exists$.
    Each quantifier is always followed by a predicate.
    \item Quantifiers bind variables. The scope of the quantifier contains all of the subformula on the right side.
    \begin{table}[ht]
        \centering 
        \begin{tabular}{|| p{2.5cm} | p{8cm} ||}
            \hline 
            formulas & explanations \\
            \hline \hline
            $\forall x. x > 0$ & $x$ is bound by the universal quantifier. \\
            \hline 
            $\forall x \exists y. x > y$ & $x$ is bound by the universal quantifier, 
              y is bound by the existential quantifier.\\
            \hline
            $\forall x \exists x. x > 0$ & $x$ is bound by the existential quantifier, 
            the rightmost quantifier binds the strongest.\\
            \hline 
        \end{tabular}
        \caption[Table 2]{Boundness of formulas}
    \end{table}
    \item Rectified formula. A formula is rectified if no variable occurs bound and free and if all quantifiers
    in the formula bind different variables. 
    \begin{table}[ht]
        \centering 
        \begin{tabular}{|| p{5cm} | p{2.5cm} ||}
            \hline 
            formulas & rectifiedness \\
            \hline \hline
            $\forall x \forall y. x + y > c$ & Yes \\
            \hline 
            $x > 0 \land \forall x . x < 42 $ & No \\
            \hline
            $\forall x. x < 42 \land \forall x. x > 13$ & No\\
            \hline 
        \end{tabular}
        \caption[Table 2]{Rectifiedness of formulas}
    \end{table}
    If you find a logical formula hard to read,
    you may convert the formula to a rectified formula. The convertion is possible by renaming the duplicated binding 
    with a new variable, which does not exist in the current formula.
    \begin{align*}
        \forall x \exists x. x > 0 &\longrightarrow \forall x \exists x_1. x_1 > 0\\
        x > 0 \land \forall x. x < 42 &\longrightarrow x > 0 \land \forall x_1. x_1 < 42\\
        \forall x. x < 42 \land \forall x. x > 13 &\longrightarrow \forall x. x < 42 \land \forall x_1. x_1 > 13
    \end{align*}
\end{enumerate}

\section*{Substitutions}
\begin{itemize}
    \item Substitutions replace \textbf{free} variables with terms. \footnote{$A[a/y]$ reads substitute y for a in $A$.}
    \item The bound variables are not substituted, for example
    \begin{align*}
        (x > 0 \land \forall x. x > 0)[1/x] \equiv 1 > 0 \land \forall x.  x > 0
    \end{align*}
    \item You cannot substitute a constant for a variable or another constant.
\end{itemize}

\section*{Problem set}
You may use this problem set to test your knowledge about the predicate logic. The problems are 
\textbf{not} directly related to the exam, but you may find them useful for understanding 
the contents.
\subsection*{Predicates and Relations}
\begin{enumerate}
    \item Let $x$ be a variable. What is the \textbf{minimal} arity of the predicate $x = y$ when y is a variable and when 
    y is not a variable?
    \item Can a predicate have an arity of \textbf{zero}? 
    \item What is the \textbf{maximum} of the arity of a predicate?
    \item What are the \textbf{reflexity, symmetry} and \textbf{transitivity} of a relation?
    \item Given an order $a \leq b \leq c$ with \textbf{transitivity}, convert it into a conjunction of relations.
    \item Let $\sim$ be a reflexive, symmetric and transitive relation over $\mathbb{N} \times \mathbb{N}$ and 
    $\circ$ be a reflexive, antisymmetric relation over $\mathbb{Z} \times \mathbb{Z}$. $\leq$ and $<$ are the ordinary less equal and less relations.
    Simplify the following formulas. For each one, determine which sets the variables must belong to so that the formula is well defined and then determine what set satisfies the formula.
    \\
    \underline{Example}: $a \sim b$ dictates that $a, b \in \mathbb{N}$ as $\sim$ is only defined over natural numbers. The formula is satisfied iff $a \sim b$, which we cannot narrow down any further since we know nothing about the $\sim$ relation.
    \begin{enumerate}
        \item $a \leq b \land b \sim a$
        \item $a < b \leq a$
        \item $a \sim b \land a \sim c \Longrightarrow c \sim b$
        \item $a \sim b \sim c \sim a \Longrightarrow c \circ b$
        \item $c \circ d \land d \circ c \equiv c \sim d \land d \sim c$
        \item $a = -3 \land a^2 \sim a \sim b \Longrightarrow a^2 \sim b$
    \end{enumerate}
\end{enumerate}
\subsection*{Quantifiers}
In the following formulas, name by which quantifier is each $x$ bound and convert them into rectified formulas.
\begin{enumerate}
    \item $\forall x \exists x \forall x. x + 1 > 0$
    \item $x > 0 \land \forall x. x < 0 \land (\exists x. x \not=0) \lor x < 0$
    \item $x > 0 \land \forall x \exists x \forall x. x < 0 \land \forall x. x > 42$
\end{enumerate}
\subsection*{Substitutions}
For each formulae $F$ given in the previous exercise, perform the substitution $F[1/x]$ and 
simplify it as far as you can. \textbf{(Do not simplify any subformulas that contain quantifiers)}
\subsection*{Advanced}
This part includes some advanced questions, which are not related to the \textbf{FPV} lecture. They 
may be inspiring if you are interested.
\begin{enumerate}
    \item Sometimes, we want to state a property or relation about a function, for example, $\forall x. f(x) > 0$. How can we elaborate
    the rectifying rule with the existence of a function?
    \item In rectified formulas, the quantifiers are also existent inside the subformulas. The formula are 
    hence less readable. Do you think of any way to improve this? You may compare the following examples.
    \begin{align*}
        &(\forall x. x = 0 \lor x \not= 0) \land \forall y. (\forall z. z > 0 \lor z < y) \land y > 0 \;\;\; \textbf{(Rectified)}\\
        &\forall x \forall y \forall z. (x = 0 \lor x \not= 0) \land (z > 0 \lor z < y) \land y > 0 \;\;\; \textbf{(Rectified Prenex)}
    \end{align*}
\end{enumerate}

\newpage 
\section*{Sample solutions}
\subsection*{Predicates and Substitutions}
\begin{enumerate}
    \item 1 if y is a variable, otherwise 2.
    \item Yes. For arity 0, we write $P$ instead of $P()$. 
    For example, $P :\equiv 1 = 1$ is a trivial property without the existence of any variable.
    \item There is no upperbound, thus $\infty$. See an example, $P(x, y) :\equiv x > 0$. 
    There is no existence of $y$ in $P$, but you can still include it. Using the similar 
    construction, we may add as many variables as we want.
    \item For a relation $\sim$
    \begin{itemize}
        \item reflexity: $a \sim a$
        \item symmetry: $a \sim b \iff b \sim a$
        \item transitivity $a \sim b \land b \sim c \Longrightarrow a \sim c$
    \end{itemize}
    The symbol of the relation does not matter, you may replace it with $\geq, \not=, \Longrightarrow$ or whatever you want/define.
    \item $a \leq b \land b \leq c \land a \leq c$
    \item 
    \begin{enumerate}
        \item Not simplifiable. $a, b \in \mathbb{N}$ for it to be defined and it holds true for $(a, b) \in \{(a, a + k) | a \in \mathbb{N} \land k \in \mathbb{N}_0 \land a \sim a + k\}$
        \item $a < b \leq a \equiv \textbf{false}\ \forall a, b \in \mathbb{Z}$
        \item The statement is equivalent to \textbf{true} $\forall a,b,c \in \mathbb{N}$:
        \begin{align*}
            &a \sim b \land a \sim c
            \overset{symmetry}{\Longrightarrow} b \sim a \land a \sim c
            \overset{transitivity}{\Longrightarrow} b \sim c
            \overset{symmetry}{\Longrightarrow} c \sim b
        \end{align*}
        \item $\forall a, b, c \in \mathbb{N}$ the statement can be simplified to $a \sim b \sim c \Longrightarrow c \circ b$, as symmetry and reflexivity already cover all the statements that the extra $\sim a$ would give us. If $\sim$ were antisymmetric, we could follow that $b = c$ and consequently $c \circ b$, but as it is, we can't say anything further. The statement therefore holds iff $\neg(a \sim b \sim c) \lor c \circ b$.
        \item The statement holds if and only if $c = d$.
        Thus, it is valid only over $(c, d) \in \{(a, a) | a \in \mathbb{N}\}$. Interestingly, this is only necessary for the right hand side (rhs) to imply the lhs. The other direction (lhs implies rhs) holds true due to antisymmetry of $\circ$ and reflexivity of $\sim$.
        \item The statement is not defined as $a = -3 \not\in \mathbb{N}$ which would be required for $a \sim b$.
    \end{enumerate}
\end{enumerate}
\subsection*{Quantifiers and Substitutions}
\begin{enumerate}
    \item Bound by the third quantifier from the left. 
    \begin{align*}
        \forall x \exists x_1 \forall x_2. x_2 + 1 > 0
    \end{align*}
    Identical after the substitution.
    \item The first and the last $x$ are free. The second is bound by the universal quantifier, and 
    the third is bound by the existential quantifier.
    \begin{align*}
        &x > 0 \land \forall x_1. x_1 < 0 \land (\exists x_2. x_2 \not= 0) \lor x < 0\\
        &(x > 0 \land \forall x_1. x_1 < 0 \land (\exists x_2. x_2 \not= 0) \lor x < 0)[1/x]\\
        \equiv&\; 1 > 0 \land \forall x_1. x_1 < 0 \land (\exists x_2. x_2 \not= 0) \lor 1 < 0\\ 
        \equiv&\;  \forall x_1. x_1 < 0 \land (\exists x_2. x_2 \not= 0)
    \end{align*}
    \item The first $x$ is free. the second is bound by the third quantifier from the left. 
    The third is bound by the fourth quantifier from the left.
    \begin{align*}
        &x > 0 \land \forall x_1 \exists x_2 \forall x_3. x_3 < 0 \land \forall x_4. x_4 > 42\\
        &(x > 0 \land \forall x_1 \exists x_2 \forall x_3. x_3 < 0 \land \forall x_4. x_4 > 42)[1/x]\\
        \equiv& \; 1 > 0 \land \forall x_1 \exists x_2 \forall x_3. x_3 < 0 \land \forall x_4. x_4 > 42 \\
        \equiv& \; \forall x_1 \exists x_2 \forall x_3. x_3 < 0 \land \forall x_4. x_4 > 42
    \end{align*}
\end{enumerate}

\subsection*{Advanced}
You may refer to the slides of the \textbf{Logic(IN2049)} lecture for the answer. The answers are accessible via 
\href{https://www21.in.tum.de/teaching/logic/SS22/assets/fol.pdf}{the link for question 1)} 
and this \href{https://www21.in.tum.de/teaching/logic/SS22/assets/normal-fol.pdf}{the link for question 2)}.

\end{document}
